% Options for packages loaded elsewhere
\PassOptionsToPackage{unicode}{hyperref}
\PassOptionsToPackage{hyphens}{url}
%
\documentclass[
]{article}
\usepackage{lmodern}
\usepackage{amssymb,amsmath}
\usepackage{ifxetex,ifluatex}
\ifnum 0\ifxetex 1\fi\ifluatex 1\fi=0 % if pdftex
  \usepackage[T1]{fontenc}
  \usepackage[utf8]{inputenc}
  \usepackage{textcomp} % provide euro and other symbols
\else % if luatex or xetex
  \usepackage{unicode-math}
  \defaultfontfeatures{Scale=MatchLowercase}
  \defaultfontfeatures[\rmfamily]{Ligatures=TeX,Scale=1}
\fi
% Use upquote if available, for straight quotes in verbatim environments
\IfFileExists{upquote.sty}{\usepackage{upquote}}{}
\IfFileExists{microtype.sty}{% use microtype if available
  \usepackage[]{microtype}
  \UseMicrotypeSet[protrusion]{basicmath} % disable protrusion for tt fonts
}{}
\makeatletter
\@ifundefined{KOMAClassName}{% if non-KOMA class
  \IfFileExists{parskip.sty}{%
    \usepackage{parskip}
  }{% else
    \setlength{\parindent}{0pt}
    \setlength{\parskip}{6pt plus 2pt minus 1pt}}
}{% if KOMA class
  \KOMAoptions{parskip=half}}
\makeatother
\usepackage{xcolor}
\IfFileExists{xurl.sty}{\usepackage{xurl}}{} % add URL line breaks if available
\IfFileExists{bookmark.sty}{\usepackage{bookmark}}{\usepackage{hyperref}}
\hypersetup{
  pdftitle={Installing GAT},
  pdfauthor={Abigail Stamm, New York State Department of Health},
  hidelinks,
  pdfcreator={LaTeX via pandoc}}
\urlstyle{same} % disable monospaced font for URLs
\usepackage[margin=1in]{geometry}
\usepackage{color}
\usepackage{fancyvrb}
\newcommand{\VerbBar}{|}
\newcommand{\VERB}{\Verb[commandchars=\\\{\}]}
\DefineVerbatimEnvironment{Highlighting}{Verbatim}{commandchars=\\\{\}}
% Add ',fontsize=\small' for more characters per line
\usepackage{framed}
\definecolor{shadecolor}{RGB}{248,248,248}
\newenvironment{Shaded}{\begin{snugshade}}{\end{snugshade}}
\newcommand{\AlertTok}[1]{\textcolor[rgb]{0.94,0.16,0.16}{#1}}
\newcommand{\AnnotationTok}[1]{\textcolor[rgb]{0.56,0.35,0.01}{\textbf{\textit{#1}}}}
\newcommand{\AttributeTok}[1]{\textcolor[rgb]{0.77,0.63,0.00}{#1}}
\newcommand{\BaseNTok}[1]{\textcolor[rgb]{0.00,0.00,0.81}{#1}}
\newcommand{\BuiltInTok}[1]{#1}
\newcommand{\CharTok}[1]{\textcolor[rgb]{0.31,0.60,0.02}{#1}}
\newcommand{\CommentTok}[1]{\textcolor[rgb]{0.56,0.35,0.01}{\textit{#1}}}
\newcommand{\CommentVarTok}[1]{\textcolor[rgb]{0.56,0.35,0.01}{\textbf{\textit{#1}}}}
\newcommand{\ConstantTok}[1]{\textcolor[rgb]{0.00,0.00,0.00}{#1}}
\newcommand{\ControlFlowTok}[1]{\textcolor[rgb]{0.13,0.29,0.53}{\textbf{#1}}}
\newcommand{\DataTypeTok}[1]{\textcolor[rgb]{0.13,0.29,0.53}{#1}}
\newcommand{\DecValTok}[1]{\textcolor[rgb]{0.00,0.00,0.81}{#1}}
\newcommand{\DocumentationTok}[1]{\textcolor[rgb]{0.56,0.35,0.01}{\textbf{\textit{#1}}}}
\newcommand{\ErrorTok}[1]{\textcolor[rgb]{0.64,0.00,0.00}{\textbf{#1}}}
\newcommand{\ExtensionTok}[1]{#1}
\newcommand{\FloatTok}[1]{\textcolor[rgb]{0.00,0.00,0.81}{#1}}
\newcommand{\FunctionTok}[1]{\textcolor[rgb]{0.00,0.00,0.00}{#1}}
\newcommand{\ImportTok}[1]{#1}
\newcommand{\InformationTok}[1]{\textcolor[rgb]{0.56,0.35,0.01}{\textbf{\textit{#1}}}}
\newcommand{\KeywordTok}[1]{\textcolor[rgb]{0.13,0.29,0.53}{\textbf{#1}}}
\newcommand{\NormalTok}[1]{#1}
\newcommand{\OperatorTok}[1]{\textcolor[rgb]{0.81,0.36,0.00}{\textbf{#1}}}
\newcommand{\OtherTok}[1]{\textcolor[rgb]{0.56,0.35,0.01}{#1}}
\newcommand{\PreprocessorTok}[1]{\textcolor[rgb]{0.56,0.35,0.01}{\textit{#1}}}
\newcommand{\RegionMarkerTok}[1]{#1}
\newcommand{\SpecialCharTok}[1]{\textcolor[rgb]{0.00,0.00,0.00}{#1}}
\newcommand{\SpecialStringTok}[1]{\textcolor[rgb]{0.31,0.60,0.02}{#1}}
\newcommand{\StringTok}[1]{\textcolor[rgb]{0.31,0.60,0.02}{#1}}
\newcommand{\VariableTok}[1]{\textcolor[rgb]{0.00,0.00,0.00}{#1}}
\newcommand{\VerbatimStringTok}[1]{\textcolor[rgb]{0.31,0.60,0.02}{#1}}
\newcommand{\WarningTok}[1]{\textcolor[rgb]{0.56,0.35,0.01}{\textbf{\textit{#1}}}}
\usepackage{graphicx,grffile}
\makeatletter
\def\maxwidth{\ifdim\Gin@nat@width>\linewidth\linewidth\else\Gin@nat@width\fi}
\def\maxheight{\ifdim\Gin@nat@height>\textheight\textheight\else\Gin@nat@height\fi}
\makeatother
% Scale images if necessary, so that they will not overflow the page
% margins by default, and it is still possible to overwrite the defaults
% using explicit options in \includegraphics[width, height, ...]{}
\setkeys{Gin}{width=\maxwidth,height=\maxheight,keepaspectratio}
% Set default figure placement to htbp
\makeatletter
\def\fps@figure{htbp}
\makeatother
\setlength{\emergencystretch}{3em} % prevent overfull lines
\providecommand{\tightlist}{%
  \setlength{\itemsep}{0pt}\setlength{\parskip}{0pt}}
\setcounter{secnumdepth}{-\maxdimen} % remove section numbering
\usepackage{fancyhdr}
\pagestyle{fancy}
\fancyfoot[CO,CE]{Draft revision \today}
\fancyfoot[LE,RO]{\thepage}
\fancypagestyle{plain}{\pagestyle{fancy}}
\fancyhead[CO,CE]{}
\fancyhead[LO,LE]{}
\fancyhead[RO,RE]{}

\title{Installing GAT}
\author{Abigail Stamm, New York State Department of Health}
\date{}

\begin{document}
\maketitle

{
\setcounter{tocdepth}{2}
\tableofcontents
}
\hypertarget{about-gat}{%
\section{About GAT}\label{about-gat}}

The Geographic Aggregation Tool (GAT) was created to simplify the
process of geographic aggregation. GAT takes a set of user-defined
parameters, aggregates based on these parameters, and outputs both the
resulting shapefiles and documents to help the user assess the quality
of the aggregation. GAT is installed and run as a package in R. Before
you can use GAT, you will need to install R.

\hypertarget{installing-r}{%
\section{Installing R}\label{installing-r}}

You can download R at \url{https://www.r-project.org/}. Click on
\emph{download R} and select any site on the list. From there, choose
your platform, then click on \emph{install R for the first time}. Next,
click on the download link to download R. If you need help installing R,
click on \emph{Installation and other instructions}.

\hypertarget{installing-rstudio-optional}{%
\section{Installing RStudio
(optional)}\label{installing-rstudio-optional}}

After you have installed R, you can install RStudio, which is a wrapper
that makes R more user-friendly and adds functionality, such as support
for Markdown and projects, and dedicated windows for environments,
tables, and images. RStudio \textbf{is not} required to run GAT.

You can download RStudio from
\url{https://rstudio.com/products/rstudio/download/}. Under ``RStudio
Desktop Open Source License Free'', click \emph{Download}. Then click on
the button, \emph{Download RStudio for Windows}. After you have
downloaded the file, install it as you would any other software program.

\hypertarget{installing-gat}{%
\section{Installing GAT}\label{installing-gat}}

GAT was compiled in R version 3.6 using RStudio and devtools. GAT has
also been tested in R versions 3.4 and 3.5. R version 3.4 or greater is
needed to install GAT.

Before installing GAT, install the necessary packages (or check that you
have them installed). One way to do this is run the following code in
your R console:

\begin{Shaded}
\begin{Highlighting}[]
\CommentTok{# necessary packages}
\NormalTok{libs <-}\StringTok{ }\KeywordTok{c}\NormalTok{(}\StringTok{"tcltk2"}\NormalTok{, }\StringTok{"rgeos"}\NormalTok{, }\StringTok{"maptools"}\NormalTok{, }\StringTok{"RColorBrewer"}\NormalTok{, }\StringTok{"classInt"}\NormalTok{, }\StringTok{"Hmisc"}\NormalTok{,}
          \StringTok{"foreign"}\NormalTok{, }\StringTok{"spdep"}\NormalTok{, }\StringTok{"plotKML"}\NormalTok{, }\StringTok{"sf"}\NormalTok{, }\StringTok{"sp"}\NormalTok{, }\StringTok{"lwgeom"}\NormalTok{, }\StringTok{"tibble"}\NormalTok{,}
          \StringTok{"graphics"}\NormalTok{, }\StringTok{"grDevices"}\NormalTok{, }\StringTok{"methods"}\NormalTok{, }\StringTok{"stats"}\NormalTok{, }\StringTok{"tcltk"}\NormalTok{, }\StringTok{"utils"}\NormalTok{, }
          \StringTok{"rgdal"}\NormalTok{)}
\CommentTok{# check if installed}
\NormalTok{req <-}\StringTok{ }\KeywordTok{unlist}\NormalTok{(}\KeywordTok{lapply}\NormalTok{(libs, require, }\DataTypeTok{character.only =} \OtherTok{TRUE}\NormalTok{))}
\NormalTok{req <-}\StringTok{ }\NormalTok{libs[}\OperatorTok{!}\NormalTok{req]}
\CommentTok{# install if not installed}
\ControlFlowTok{if}\NormalTok{ (}\KeywordTok{length}\NormalTok{(req) }\OperatorTok{>}\StringTok{ }\DecValTok{0}\NormalTok{) \{}
  \ControlFlowTok{for}\NormalTok{ (i }\ControlFlowTok{in} \DecValTok{1}\OperatorTok{:}\KeywordTok{length}\NormalTok{(req)) \{}
    \KeywordTok{install.packages}\NormalTok{(req[i])}
\NormalTok{  \}}
  \KeywordTok{rm}\NormalTok{(i)}
\NormalTok{\}}
\KeywordTok{rm}\NormalTok{(req, libs)}
\end{Highlighting}
\end{Shaded}

Also check whether you have the XML package installed. If you do not or
are not sure, this code will install it in Windows with R version 3.6.x:

\begin{Shaded}
\begin{Highlighting}[]
\NormalTok{pkg_link =}\StringTok{ "https://cran.r-project.org/bin/windows/contrib/4.0/XML_3.99-0.5.zip"}
\KeywordTok{install.packages}\NormalTok{(pkg_link, }\DataTypeTok{type =} \StringTok{"binary"}\NormalTok{, }\DataTypeTok{repos =} \OtherTok{NULL}\NormalTok{)}
\end{Highlighting}
\end{Shaded}

Save the zip file for GAT. It will have a name like
``gatpkg\_\textless ver\textgreater.tar.gz'' where
\textless ver\textgreater{} will be the package version number.

If you prefer to install GAT through base R (instead of RStudio), click
on ``Packages'' \textgreater{} ``Install package(s) from local
files\ldots{}''. Navigate to the gatpkg file and click ``Open''. Wait a
few minutes while GAT installs.

If you prefer to install GAT through RStudio, click on ``Tools''
\textgreater{} ``Install Packages\ldots{}''. In the ``Install from:''
drop-down, select ``Package Archive File''. A dialog should open
immediately to select the ``gatpkg\_\textless ver\textgreater.tar.gz''
file, but if it does not, click the ``Browse\ldots{}'' button. Navigate
to the gatpkg file and click ``Install''. Wait a few minutes while GAT
installs.

\hypertarget{using-gat}{%
\section{Using GAT}\label{using-gat}}

If you have not used GAT before, follow the tutorial vignette embedded
in \texttt{gatpkg} to learn how GAT works. Access the tutorial by
running the code,

\begin{Shaded}
\begin{Highlighting}[]
\KeywordTok{browseVignettes}\NormalTok{(}\StringTok{"gatpkg"}\NormalTok{)}
\end{Highlighting}
\end{Shaded}

Then click on the tutorial vignette HTML link. This will walk you
through many of the options and settings available in GAT using
shapefiles embedded in \texttt{gatpkg}.

The other vignettes embedded in the package include:

\begin{itemize}
\tightlist
\item
  \textbf{Technical Notes}: How each merge type works, compactness
  ratio, thinning
\item
  \textbf{Setting up GAT}: A shorter version of this document
\item
  \textbf{Shapefile specifications}: Requirements for a shapefile you
  plan to process with GAT
\item
  \textbf{Evaluating results}: Ways to identify and address issues with
  your aggregated areas
\item
  \textbf{Troubleshooting}: How to handle issues that arise in GAT
\item
  \textbf{Change Log}: Changes across GAT versions, notably from the
  2015 script to the 2020 package
\end{itemize}

To run the default version of GAT, you only need one line of code,

\begin{Shaded}
\begin{Highlighting}[]
\NormalTok{gatpkg}\OperatorTok{::}\KeywordTok{runGATprogram}\NormalTok{()}
\end{Highlighting}
\end{Shaded}

To learn about ways to customize GAT, check the technical notes or run
the code,

\begin{Shaded}
\begin{Highlighting}[]
\NormalTok{?gatpkg}\OperatorTok{::}\NormalTok{runGATprogram}
\end{Highlighting}
\end{Shaded}

If you have used R before, \texttt{gatpkg} contains over 30 custom
functions that can be modified or combined in different ways to meet
your specific needs. Each of these functions has a corresponding help
file that provides options and simple examples.

\hypertarget{uninstalling-gat}{%
\section{Uninstalling GAT}\label{uninstalling-gat}}

To uninstall \texttt{gatpkg}, run the code,

\begin{Shaded}
\begin{Highlighting}[]
\KeywordTok{remove.packages}\NormalTok{(}\StringTok{"gatpkg"}\NormalTok{)}
\end{Highlighting}
\end{Shaded}

\end{document}
